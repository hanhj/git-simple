\documentclass[fleqn]{article}
%\usepackage{ctex}
\usepackage{xeCJK}
\setCJKmainfont{AR PL UMing TW MBE}
\begin{document}
\title{高数学习}
\author{韩海舰}
\maketitle
\setlength{\mathindent}{10pt}
\begin{flushleft}
\section{第一章\quad 函数,极限,连续性}
\subsection{第一周练习}
1.已知f(x+1)的定义域为[0,a](a>0),则f(x)的定义域为()。
\begin{enumerate}
\item $[1,a+1]$
\item $[-1,a+1]$
\item $[a,a+1]$
\item $[a-1,a]$
\end{enumerate}

正确答案:A你错选为B

2. \[\texttt{若}\lim_{n\to \infty}x_n=a\texttt{(a为常数),则下列说法不正确的是。}\]
\begin{enumerate}
\item 数列$\{x_n\}$有界。
\item \[\lim_{n\to \infty}x_{2n}=a\]
\item 若$x_n>0$(n=1,2...n),则a>0
\item 常数a唯一。
\end{enumerate}
正确答案:C你没选择任何选项

极限的性质包括:唯一性,有界性,保号性。其中保号性是指如果极限>0,则$x_n$>0。
\subsection{第二周练习}
\par
1.已知函数\[
f(x)=\left\{\begin{array}{ll}
x^2sin\frac{1}{x} ,& x<0 \\
\frac{\sqrt{1+x^2}-1}{x}, & x>0
\end{array}
\right. 
\]
\[\texttt{则}\lim_{x\to 0}f(x)=(?)
\]
结果为0
\par
2.\[
\texttt{若}\lim_{x\to 0}f(x)=0,\texttt{则()}\]
\begin{enumerate}
\item \[\texttt{仅当}\lim_{x\to x_0}g(x)=0\texttt{时,才有}\lim_{x\to x_0}f(x)g(x)=0\texttt{成立。}\]
\item \[\texttt{当}g(x)\texttt{为任意函数时,有}\lim_{x\to x_0}f(x)g(x)=0\texttt{成立}\]
\item \[\texttt{仅当}g(x)\texttt{为常数时,才能使}\lim_{x\to x_0}f(x)g(x)=0\texttt{成立}\]
\item \[\texttt{当}g(x)\texttt{有界时,能使}\lim_{x\to x_0}f(x)g(x)=0\texttt{成立}\]
\end{enumerate}
答案是4.\[\lim_{x\to 0}f(x)=0\]是无穷小,无穷小与有界函数之积是无穷小。

3.\[
\lim_{x\to 0}(1-x)^{\frac{1}{sinx}}=(?)\]
\begin{enumerate}
\item 1
\item e
\item $e^{-1}$
\item $e^{-2}$
\end{enumerate}

正确答案:C你错选为B


\subsection{第三周练习}
等价无穷小:
\[
\begin{array}{ll}
\ln(x+1)\sim x ,& 1-cos(x) \sim \frac{1}{2}x^2\\
sin(x) \sim x ,& tan(x) \sim x\\
e^x-1 \sim x, & a^x -1 \sim x\ln a
\end{array}
\]
有用极限:
\[
\begin{array}{ll}
\lim_{x\to 0}(1+x)^{\frac{1}{x}}=e
\end{array}
\]

无穷小的关系:
\[
\alpha =lim{f(x)}=0,\beta =\lim{g(x)}=0,\texttt{均是无穷小}
\]
\[
\begin{array}{|l|l|l|}
\hline
\texttt{高阶无穷小:}&\frac{\alpha}{\beta}=0&\alpha\texttt{是}\beta\texttt{的高阶无穷小}\\
\hline
\texttt{等价无穷小:}&\frac{\alpha}{\beta}=1&\alpha\texttt{是}\beta\texttt{的等阶无穷小}\\
\hline
\texttt{同价无穷小:}&\frac{\alpha}{\beta}=c\quad(c\neq 0)&\alpha\texttt{是}\beta\texttt{的同阶无穷小}\\
\hline
\texttt{k价无穷小:}&\frac{\alpha}{\beta^k}=c\quad(c\neq 0)&\alpha\texttt{是}\beta\texttt{的k阶无穷小}\\
\hline
\end{array}
\]

\section{导数与微分}
\subsection{第四周练习}
1.设函数f(u)可导,且$y=f(x^2)$当自变量x在x=1处取得增量$\Delta x=-0.1$时,相应的函数增量$\Delta y$的线性主部为0.1,则$f^{'}(1)=().$
\begin{enumerate}
\item 0.1
\item 1
\item -0.5
\item -1
\end{enumerate}
正确答案:C你错选为D

2.函数y=f(x)在$x_0$处连续、可导、可微的关系中不正确的是:
\begin{enumerate}
\item 可导是可微的充分必要条件
\item 可微是连续的充分条件
\item 连续是可导的充分必要条件
\item 连续式可微的必要条件
\end{enumerate}
正确答案:C你错选为A
\par
连续:
\begin{enumerate}
\item f(x)在$x_0$处有定义
\item f(x)在$x_0$处有极限
\item \[\lim_{x\to x0}f(x)=f(x0)\]
\end{enumerate}
可导:
\begin{enumerate}
\item f(x)在$x_0$的邻域内有定义
\item \[\lim_{\Delta x\to 0}\frac{\Delta y}{\Delta x}=\lim_{\Delta x\to 0}\frac{f(x0+\Delta x)-f(x_0)}{\Delta x}\]存在
\item \[f_{+}^{'}(x_0)=f_{-}^{'}(x_0)\]
\end{enumerate}
可导$\Longrightarrow$连续,但连续未必可导。可导
$\Longleftrightarrow$可微

二阶导数:\[\frac{d^2y}{dx^2}=\frac{d\frac{dy}{dx}}{dx}\]

\section{第三章微分中值定理与导数应用}
\subsection{第六周练习}
1.求\[\lim_{x\to 0}\left\lbrace\frac{sinx}{x}\right\rbrace^\frac{1}{1-cosx}=()\]
\begin{enumerate}
\item 1
\item $e^{-\frac{1}{3}}$
\item $e^{\frac{1}{6}}$
\item $e^2$
\end{enumerate}
正确答案:B你错选为A

\paragraph
拐点:凹凸转折点。\\
拐点的充分必要条件:f(x)在(a,b)内二阶可导,$f^{''}(x_0)=0,x_0\in{(a,b)}$,而且$f^{''}(x_0)$处两边符号不相等

\paragraph
曲率:反应曲线的弯曲程度。直线的曲率为0,圆的曲率为1/R.圆的半径越大,则曲率越小,否则曲率越大。
\[K=\lim_{\Delta S\to 0}\frac{\alpha}{\Delta S} \]$\alpha$是弧的转角。
曲率的计算公式:\[\frac{|y^{''}|}{(1+y^{' 2})^{\frac{3}{2}}}\]

\begin{verbatim}
matlab中的函数
定义符号:syms x
求导:diff(f)
例如:f=sin(x);diff(f);
积分:
符号积分:
int(fun,x)计算不定积分
int(fun,x,a,b):计算定积分
例如:
f=sinx;int(f)
int(f,-pi,pi);

数值积分:
trapz(x,y):梯形积分
例如:x=-1:0.01:1
y=tan(x);
trapz(x,y);

quad(fun,x,a,b)辛普森积分
例如:
y=@(x)tan(x)
quad(y,-1,1)

pretty(fx)人性化显示公式
\end{verbatim}

\section{第四章 一元函数积分学}
\subsection{第九周练习}
1.f(x)在[a,b]上连续是$\int_a^b{f(x)dx}$的()
\begin{enumerate}
\item 充分必要条件
\item 必要非充分条件
\item 既非充分也非必要条件
\item 充分而非必要条件
\end{enumerate}
正确答案:D你错选为C
\paragraph
2.求$\int \frac{\sqrt{x^2-9}}{x^2}dx$\\
正确答案:$ln|x+\sqrt{x^2-9}|-\frac{1}{x}\sqrt{x^2-9}+C$
\paragraph
3. 求$\int_a^bf(mx+n)dx=$\\
正确答案:$\frac{1}{m}\int_{ma+n}^{mb+n}f(x)ds$
\paragraph
4 求 $\int_0 ^{1/2} \frac{x^2}{\sqrt{1-x^2}}dx$\\
正确答案:$\pi/12-\sqrt{3}/8$
\paragraph
5求$\int_{-2}^{2}(\exp{x^2}sinx^3-\sqrt{4-x^2})dx$\\
正确答案:$-2\pi$
\end{flushleft}
\end{document}

