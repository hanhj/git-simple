\documentclass{article}
\usepackage{CJK}
\usepackage{amssymb}
\begin{document}
\begin{CJK}{UTF8}{gbsn}
\title{根据国网检测大纲整理的故障指示器功能}
\author{hanhj}
\maketitle
根据国网检测大纲2018故障指示器的功能整理如下:
检测大纲中故障指示器按照工作原理分成两种,一种是传统型,一种是暂态录波型。
\section{传统型功能要求}
\begin{enumerate}
\item	短路故障检测
\par
	\textbf{要求:}\\当线路发生短路故障时,故障指示器应能判断出故障类型(瞬时性或永久性故障),并指示。\\
	\textbf{解读:}\\此条要求故指能够在线路发生短路故障时,能够判别出当前故障是瞬时性还是永久性故障,并指示。\\
	瞬时性或永久性故障的判断依据是当线路失电后,如果在设定时间内恢复供电,则为瞬时性故障,否则为永久性故障。\\
	判断失电依据:电流低于定值,电压低于定值。\\
	\textbf{原理:}\\
	\footnote{$I_x$是任意相电流,$U_x$是场强}
	失电判据:\\
	\[
		\left\{ 
			\begin{array}{ll}
				I_x<dz\\
				U_x<dz 
			\end{array}
		\right.
	\]
	故障性质判断:\\
		当检测到线路失电,启动计时,时间到,如果此时恢复供电则为瞬时性故障,否则为永久性故障。			

\item	自动检测故障
\par
	\textbf{要求:}\\应自适应负荷电流大小,当检测到线路电流突变,突变电流持续一段时间后,各相电场强度大幅下降,且残余电流不超过5A,应能就地采集故障信息,就地指示故障,且能将将故障信息上传到主站。\\
	\textbf{解读:}\\此条要求实际是要求故指能判断线路失电故障。即断路器在线路故障时,跳开,此时线路将失电,此时要求故指能够检测这种状况,与上面的失电判断原理一致,只是加上时间要求。\\
	\textbf{原理:}
	\footnote{$I_x$是任意相电流,$I_z$是残余电流,$U_x$是场强。}
	\begin{equation}	
		\left\{ \begin{array}{ll}
				\Delta I_x>I_{dz} & \textrm{启动条件}\\
			    t>T_{dz} & \textrm{突变电流持续的时间}\\
				I_z<5A & \textrm{持续时间后,剩余电流}\\ 
			\Delta U_x>U_{dz}
		\end{array}
		\right .
	\end{equation}
\item	接地故障检测和告警
	\par
	\textbf{要求:}\\当线路发生接地故障时,故指应能够以外施加信号检测法,暂态特征检测法,稳态特征检测法等方式检测接地故障。\\
	\textbf{解读:}	\\外施加信号和稳态特征就是检测$I_0$的大小,暂态特征参见暂态型说明。\\
	\textbf{外施加和稳态特征识别原理:}\\
	\begin{equation}
		\left\{
			\begin{array}{ll}
				I_0>I_{dz} & \textrm{零序电流大小}\\
				  t>T_{dz} & \textrm{持续时间}
			\end{array}
			\right.
	\end{equation}
\item	故障后复归
	\par
	\textbf{要求:}\\
	架空型应能在规定时间或线路恢复正常供电后自动复位,也可根据故障性质(永久性或瞬时性)自动选择复归方式。\\
	电缆型应能在手动、规定时间或线路恢复供电自动复归。也可根据故障性质自动选择复归方式。\\
	\textbf{解读:}\\
	当检测到失电故障后,可以判断出故障性质(瞬时性或永久性故障。接地故障还是相间短路故障并不能准确判断,因为零序电流过大,也有可能导致负荷电流过大。此外,要求中也不要求检测相间短路故障,参见第1,2条要求,原因是当短路故障时,断路器肯定要跳开,而接地故障时,断路器可能不会跳开)。此时终端要指示故障,那么何时消除故障,则有几种方式。\\
	1)按照时间消除:即从检测出失电故障开始计时,等到一定时间后即消除指示。\\	2) 线路恢复供电后消除:即一旦检测到恢复供电即消除。\\
	3) 手动消除。\\
	以上几种方式,不能自动切换,即设定什么方式就是什么方式。为了利用所检测到的故障性质,还有一种自动方式,即按照故障性质来消除。可以这样来设定:当是瞬时性故障时,按照时间复归;当是永久性故障时,按照自动恢复供电复归。手动复归方式优先于前两种方式。\\
	\textbf{原理:}\\
	略

\item 低电量告警
	\par
	\textbf{要求:}\\
	架空型采集单元应能以翻牌锁死方式指示电池低电量。电缆型采集单元、显示面板均以变化色卡颜色指示低电量。\\
	\textbf{解读:}\\
	这个没有什么好说的,就是检测电池电压,加上机械动作。\\
	\textbf{原理:}\\
	略
\item 防误动
	\par
	\textbf{要求:}\\
	负荷波动不应误报警;变压器空载合闸涌流不应误报警;线路突合负载涌流不应误报警;人工投切大负荷不应误报警;非故障相重合闸涌流不应误报警。\\
	\textbf{解读:}\\
	这些要求是一些原则性要求,具体到实现,对于相间短路故障,由于不要求检测故障本身,而只是检测断路器跳闸后的失电情况,因此以上情况并不存在。所存在的问题在于当接地故障时,有可能导致零序电流过大,此时可以通过延时来解决。具体来说,当检测到线路由失电转换到来电状态时,可以等待一段时间,再进行检测。\\
	\textbf{原理:}\\
	略
\item 重合闸识别
	\par
	\textbf{要求:}\\
	1)应能识别间隔为0.2s的瞬时性故障,并正确动作;\\
	2) 非故障线路安装的故指经受0.2s重合闸间隔停电后,在感受到重合闸涌流后不应误动作。\\
	\textbf{解读:}\\
	第一个要求实际上是对检测时间分辨率的要求,即至少能够分辨到0.2s;第二个要求与上面的防误动要求一致。具体实现方法同上。\\
	\textbf{原理:}\\
	略
\item 监测和管理
	\par
	\textbf{要求:}\\
	1)汇集单元至少能够满足3条线路(每条线路3只)采集单元的接入要求。\\
	2) 应具备历史数据存储能力,包括不少于256条的事件顺序记录,30条本地操作记录,10条装置异常记录。\\
	3) 应具备本地及远方维护功能,支持远方程序升级。\\
	\textbf{解读:}\\
	第一条实际上是对硬件cpu运算速度要求,要求能够实时采集至少9个采集单元信息,并进行计算。第二条是对硬件存储容量的要求。第三条是要求具备远程升级功能。\\
	\textbf{原理:}\\
	这个没有什么好说的。
\item 带电装卸
	\par
	\textbf{要求:}\\
	架空型应具有带电装卸功能,装卸过程中不应误报警。\\
	\textbf{解读:}\\
	这个没有什么好说的。\\
	\textbf{原理:}\\
	略
	
\end{enumerate}
\section{暂态录波型功能要求}


相关文件:'F:/document/其他资料/规范性引用文件/国网规范/配电自动化/配电自动化招标技术规范/专业检测2017/故障指示器'
\end{CJK}
\end{document}

