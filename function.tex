\documentclass{article}
\usepackage{CJK}
\usepackage{amssymb}
\begin{document}
\begin{CJK}{UTF8}{gbsn}
\title{根据国网检测大纲整理的故障指示器功能}
\author{hanhj}
\maketitle
根据国网检测大纲2017故障指示器的功能整理如下:
检测大纲中故障指示器按照工作原理分成两种,一种是传统型,一种是暂态录波型。
\section{传统型功能要求}
\begin{enumerate}
\item	短路故障检测
\par
	\textbf{要求:}\\当线路发生短路故障时,故障指示器应能判断出故障类型(瞬时性或永久性故障),并指示。\\
	\textbf{解读:}\\此条要求故指能够在线路发生短路故障时,能够判别出当前故障是瞬时性还是永久性故障,并指示。\\
	瞬时性或永久性故障的判断依据是当线路失电后,如果在设定时间内恢复供电,则为瞬时性故障,否则为永久性故障。\\
	判断失电依据:电流低于定值,电压低于定值。\\
	\textbf{原理:}\\
	\footnote{$I_x$是任意相电流,$U_x$是场强}
	失电判据:\\
	\[
		\left\{ 
			\begin{array}{ll}
				I_x<dz\\
				U_x<dz 
			\end{array}
		\right.
	\]
	故障性质判断:\\
		当检测到线路失电,启动计时,时间到,如果此时恢复供电则为瞬时性故障,否则为永久性故障。			

\item	自动检测故障
\par
	\textbf{要求:}\\应自适应负荷电流大小,当检测到线路电流突变,突变电流持续一段时间后,各相电场强度大幅下降,且残余电流不超过5A,应能就地采集故障信息,就地指示故障,且能将将故障信息上传到主站。\\
	\textbf{解读:}\\此条要求实际是要求故指能判断线路失电故障。即断路器在线路故障时,跳开,此时线路将失电,此时要求故指能够检测这种状况,与上面的失电判断原理一致,只是加上时间要求。\\
	\textbf{原理:}
	\footnote{$I_x$是任意相电流,$I_z$是残余电流,$U_x$是场强。}
	\begin{equation}	
		\left\{ \begin{array}{ll}
				\Delta I_x>I_{dz} & \textrm{启动条件}\\
			    t>T_{dz} & \textrm{突变电流持续的时间}\\
				I_z<5A & \textrm{持续时间后,剩余电流}\\ 
			\Delta U_x>U_{dz}
		\end{array}
		\right .
	\end{equation}
\item	接地故障检测和告警
	\par
	\textbf{要求:}\\当线路发生接地故障时,故指应能够以外施加信号检测法,暂态特征检测法,稳态特征检测法等方式检测接地故障。\\
	\textbf{解读:}	\\外施加信号和稳态特征就是检测$I_0$的大小,暂态特征参见暂态型说明。\\
	\textbf{外施加和稳态特征识别原理:}\\
	\begin{equation}
		\left\{
			\begin{array}{ll}
				I_0>I_{dz} & \textrm{零序电流大小}\\
				  t>T_{dz} & \textrm{持续时间}
			\end{array}
			\right.
	\end{equation}
\item	故障后复归
	\par
	\textbf{要求:}\\
	架空型应能在规定时间或线路恢复正常供电后自动复位,也可根据故障性质(永久性或瞬时性)自动选择复归方式。\\
	电缆型应能在手动、规定时间或线路恢复供电自动复归。也可根据故障性质自动选择复归方式。\\
	\textbf{解读:}\\
	当检测到失电故障后,可以判断出故障性质(瞬时性或永久性故障。接地故障还是相间短路故障并不能准确判断,因为零序电流过大,也有可能导致负荷电流过大。此外,要求中也不要求检测相间短路故障,参见第1,2条要求,原因是当短路故障时,断路器肯定要跳开,而接地故障时,断路器可能不会跳开)。此时终端要指示故障,那么何时消除故障,则有几种方式。\\
	1)按照时间消除:即从检测出失电故障开始计时,等到一定时间后即消除指示。\\	2) 线路恢复供电后消除:即一旦检测到恢复供电即消除。\\
	3) 手动消除。\\
	以上几种方式,不能自动切换,即设定什么方式就是什么方式。为了利用所检测到的故障性质,还有一种自动方式,即按照故障性质来消除。可以这样来设定:当是瞬时性故障时,按照时间复归;当是永久性故障时,按照自动恢复供电复归。手动复归方式优先于前两种方式。\\
	\textbf{原理:}\\
	略

\item 低电量告警
	\par
	\textbf{要求:}\\
	架空型采集单元应能以翻牌锁死方式指示电池低电量。电缆型采集单元、显示面板均以变化色卡颜色指示低电量。\\
	\textbf{解读:}\\
	这个没有什么好说的,就是检测电池电压,加上机械动作。\\
	\textbf{原理:}\\
	略
\item 防误动
	\par
	\textbf{要求:}\\
	负荷波动不应误报警;变压器空载合闸涌流不应误报警;线路突合负载涌流不应误报警;人工投切大负荷不应误报警;非故障相重合闸涌流不应误报警。\\
	\textbf{解读:}\\
	这些要求是一些原则性要求,具体到实现,对于相间短路故障,由于不要求检测故障本身,而只是检测断路器跳闸后的失电情况,因此以上情况并不存在。所存在的问题在于当接地故障时,有可能导致零序电流过大,此时可以通过延时来解决。具体来说,当检测到线路由失电转换到来电状态时,可以等待一段时间,再进行检测。\\
	\textbf{原理:}\\
	略
\item 重合闸识别
	\par
	\textbf{要求:}\\
	1)应能识别间隔为0.2s的瞬时性故障,并正确动作;\\
	2) 非故障线路安装的故指经受0.2s重合闸间隔停电后,在感受到重合闸涌流后不应误动作。\\
	\textbf{解读:}\\
	第一个要求实际上是对检测时间分辨率的要求,即至少能够分辨到0.2s;第二个要求与上面的防误动要求一致。具体实现方法同上。\\
	\textbf{原理:}\\
	略
\item 监测和管理
	\par
	\textbf{要求:}\\
	1)汇集单元至少能够满足3条线路(每条线路3只)采集单元的接入要求。\\
	2) 应具备历史数据存储能力,包括不少于256条的事件顺序记录,30条本地操作记录,10条装置异常记录。\\
	3) 应具备本地及远方维护功能,支持远方程序升级。\\
	\textbf{解读:}\\
	第一条实际上是对硬件cpu运算速度要求,要求能够实时采集至少9个采集单元信息,并进行计算。第二条是对硬件存储容量的要求。第三条是要求具备远程升级功能。\\
	\textbf{原理:}\\
	这个没有什么好说的。
\item 带电装卸
	\par
	\textbf{要求:}\\
	架空型应具有带电装卸功能,装卸过程中不应误报警。\\
	\textbf{解读:}\\
	这个没有什么好说的。\\
	\textbf{原理:}\\
	略
\item 通讯功能
	\par
	\textbf{要求:}\\
	1)应能通过无线通信方式主动上送告警信息,复归信息以及监测的负荷电流,故障数据等信息至配电主站,故障信息上送至主站时间应小于60s,并支持主站召测全数据功能。\\
	2)具备对时功能,接收主站或其他时间同步装置的对时命令,与系统时钟保持同步。守时精度2s/24h.\\
	3)当后备电池电压降低到阈值时,应将其状态上传至主站,也可根据需要进行本地报警。当外部电源失去时,后备电源应能自动无缝投入,且保证将失电前的完整的故障数据上传至主站。\\
	4)采集单元与汇集单元之间应能以无线、光纤灯方式进行通讯。无线方式宜采用微功率方式。\\
	5)汇集单元应适应无线传输要求,在网络中断后,具有本地存储和调用模式,保存故障信息等关键数据。\\
	6)汇集单元可以通过实时在线或准实时在线方式与配电主站通信,并能以不大于24h的间隔上传负荷曲线数据至主站。\\
	\textbf{解读:}\\
	1)这个实际是对规约以及数据项的要求。具体来说支持101,104规约,支持101平衡方式的主动上送数据或非平衡方式的事件收集方式上送事件;支持104规约中的主动上送数据。支持总召唤。数据项包括遥信数据,包括各种告警信息,复归信息。遥测数据:包括负荷数据,支持主动上送变化负荷遥测数据。\\
	2)这个实际上是要求实现101,104规约中的对时功能。守时精度是对硬件的要求。要采用高精度的时钟芯片,并及时调整终端时钟。\\
	3)这个要求遥信数据项中应包括电池低电量告警。后面的要求是对硬件的要求。\\
	4)这个是对通讯方式的选择,在硬件实现时要注意。\\
	5)这个要求是在网络中断后,应当保存未上送的遥信数据,一旦通讯恢复能够上送这些信息。保存可以是保存在内存中,也可以保存在磁盘中。\\
	6)这个包含两个要求。一个是要求保存负荷曲线数据(96点?),一个是能够至少以24h为间隔,上送该数据。前者包括了对存储容量的要求(至少96点的负荷数据,至少支持9个采集单元),数据保存功能实现。后者是要求规约实现负荷曲线上送功能,即文件传输功能。
\item 电气性能
	\par
	\textbf{要求:}\\
	1)短路故障报警启动误差不超过$\pm$10\%\\
	2)最小可识别故障电流持续时间不应大于40ms。\\
	3)电缆型故指电缆温度测量误差不大于3°C。\\
	4)低电量告警电压误差不大于$\pm$2\%\\
	5)负荷电流误差负荷以下要求:
	\begin{itemize}
			\item $0\leq I < 100$时,误差为$\pm$3A.
			\item$100\leq I < 600$时,误差为$\pm$3\%。
	\end{itemize}
	6)上电自动复位时间小于5min。定时复位时间可设定,设定范围小于48h,最小分辨率1min,定时复位时间误差不超过$\pm$1\%。\\
	7)接地故障识别率:
	\begin{itemize}
			\item 金属性接地达到100\%
			\item 小电阻接地达到100\%
			\item 弧光接地达到90\%
			\item 高阻接地(800$\Omega$以下)达到90\%
	\end{itemize}
	\textbf{解读:}\\
	1,3,4,5是对精度的要求。2是对故障检测时间的要求,即保护计算频率应小于40ms。通常来说,计算的频率是1个周波即20ms。但是目前不知道该计算是在采集器还是在汇集器,如果在汇集器则通讯频率至少要在20ms内,通讯可靠性以及通讯频率会很高。最好在采集器中计算,汇集器收集告警遥信。6是对复位时间的要求,包括时间精度(最小分辨率,时间误差),功能(定时复位时间可设,上电自动复位时间小于5min)。\\上电自动复位时间,即当检测到失电$->$来电事件后,为了防止上文所说的误动情况,需要等待一段时间,该时间不应大于5min。7是对可靠性的要求,需要通过实验来验证。\\
	\textbf{原理:}\\
	略
\item 外观 
	\par
	\textbf{要求:}\\
	具体要求参见检测大纲,比较明了,这里不赘述。

\item 绝缘性能
	\par
	\textbf{要求:}\\
	绝缘电阻:电杆固定安装的汇集单元电源回路与外壳之间的绝缘电阻$\geq 5M\Omega$(使用250V绝缘电阻表,绝缘电压$U_i\leq 60V$)。\\
	绝缘强度:电杆固定安装的汇集单元电源回路与外壳之间的额定绝缘电压$U_i \leq 60V$时,施加500V工频电压无击穿,无闪络。\\
	\textbf{解读}\\
	没什么好说的。\\
	\textbf{原理:}\\
	略。

\item 低温性能
	\par
	要求同电气性能。
\item 高温性能
	\par
	要求同电气性能
\item 盐雾实验
	\par
	\textbf{要求:}\\
	在盐雾实验后,外观应无损坏,功能应正常。\\
	\textbf{解读:}\\
	主要是要求材质合格。
\item 跌落实验
	\par
	\textbf{要求:}\\
	跌落实验后,外观应无损坏,功能应正常。\\
	\textbf{解读:}\\
	主要是要求材质合格。
\item 卡线握力实验
	\par
	\textbf{要求:}\\
	参见检测大纲。\\
	\textbf{解读}\\
	没什么好说的,参见要求。\\
	\textbf{原理:}\\
	略。

\item 射频磁场实验
	\par
	\textbf{要求:}\\
	参见检测大纲。\\
	\textbf{解读}\\
	没什么好说的,参见要求。\\
	\textbf{原理:}\\
	略。

\item 浪涌实验
	\par
	\textbf{要求:}\\
	参见检测大纲。\\
	\textbf{解读}\\
	没什么好说的,参见要求。\\
	\textbf{原理:}\\
	略。

\item 脉冲群实验
	\par
	\textbf{要求:}\\
	参见检测大纲。\\
	\textbf{解读}\\
	没什么好说的,参见要求。\\
	\textbf{原理:}\\
	略。

\item 阻尼磁场实验
	\par
	\textbf{要求:}\\
	参见检测大纲。\\
	\textbf{解读}\\
	没什么好说的,参见要求。\\
	\textbf{原理:}\\
	略。

\item 过量实验
	\par
	\textbf{要求:}\\
	1)可承受10KV,20kA的短路电流,2s,外观应无损坏,功能应正常。\\
	2)可承受35KV,31.5kA的短路电流,4s,外观应无损坏,功能应正常。\\
	\textbf{解读:}\\
	这个是对互感器的要求。

\item 临近干扰实验	
	\par
	\textbf{要求:}\\
	1)当相邻300m线路发生故障时,本线路不应发出误告警\\
	2)本线路发生故障时,相邻300m的导线不应发出本线路正常告警。\\
	\textbf{解读:}\\
	第一条要求即如果有两条线路1,2,假设线路1发生短路或接地故障,线路2没有故障,此时线路2应当没有短路电流或零序电流,此时线路2应当不会发出告警信号。该项要求其实是对可靠性的要求,即不会因相邻线路(300m)出现故障,而导致出现本线路误报。但是终端检测故障实际上是对本线路的电信号进行监测,并不知道是否有相邻线路存在,如果由于线路1的故障导致线路2的电信号也呈现故障特征,则无法识别。所以本条要求并不是很合理,只是要求终端能够抗干扰信号,对于由于故障线路所产生的例如涌流,脉冲群等干扰信号不至于发生误报警。\\
	第二条要求实际上是对通讯可靠性的要求,因为故障信号的采集来自于采集单元,然后发送到汇集单元,如果发生故障时,线路距离较远(300m),有可能由于故障信号所引发的干扰信号导致通讯出现异常,进而导致不能正确采集信号,而导致不能告警。
\\
所以这两条要求实际上是在终端满足前面常规模拟干扰实验后的再次实际检验。
\item 着火实验
	\par
	\textbf{要求:}\\
	采集单元和架空线悬挂的汇集单元外壳应当采用非金属阻燃材料,能够承受5级着火危险。\\
	\textbf{解读:}\\
	本条要求就是对材质的要求。没什么好说的。
\item 电源及功率消耗实验
	\par
	\textbf{要求:}\\
	1)线路负荷电流不小于10A时,TA取电5s内应能满足全部功能要求。\\
	2)采集单元非充电电池单独供电时,最小工作电流应不大于40uA。\\
	3)采用太阳能供电的汇集单元电池充满电后额定电压不低于DC12V。采用TA取电的汇集单元电池额定电压不低于DC3.6V。\\
	4)就地型故指采集单元、显示面板静态功耗应小于15uA;远传型故指采集单元功耗应小于40uA,汇集单元整机正常工作功耗应不大于5VA。\\
	\textbf{解读:}\\
	总结一下:\\
	取电方式可以有TA取电,太阳能取电,电池供电。这里没有PT取电,因为不可能在终端上再加装PT。采集器一般采用TA取电,加上小型的后备电池(因为需要翻牌指示)。汇集器一般采用TA取电或太阳能取电,如果用TA取电要在加上后备电池,用太阳能一般配置了后备电池。由于受到体积限制,对于采集器后备电池不可能太大,对于汇集器来说,后备电池可以稍大。\\
	关于功耗的要求包括工作电流,整机功耗,额定电压。总的来说要求是低功耗,因为没有PT供电,对于采集器,如果采用TA取电,负荷电流有可能有,当失电时没有,此时要靠后备电池支持翻牌指示,而停电时间可能很长(48h,巡线时间),所以要求采集器低功耗。对于汇集器而言,如果采用TA取电,基于上面同样的原因,同样需要低功耗。如果汇集器采用太阳能取电,并不能保证天气良好,所以同样需要低功耗。\\
	对采集器和汇集器有不同的功耗要求。\\
	采集器:
	\begin{itemize}
			\item 非充电电池供电时,最小工作电流不大于40uA。即后备电池是一次性的要求功耗更小。
			\item 就地型采集单元,最小工作电流不大于15uA;远传型采集单元工作电流不大于40uA。就地型是否就是就地有翻牌指示,远传型不就地指示?是否就地型功耗要大一些。
	\end{itemize}
	汇集器:
	\begin{itemize}
			\item 采用太阳能取电的额定电压不小于12V,采用TA取电的额定电压不小于3.6V。这条要求对于额定电压也有限制,防止电压过低,不能正常工作。
			\item 整机功耗不大于5VA。
	\end{itemize}
	除此之外,第一条针对TA取电的,实际上假设开始后备电池没电,此时TA送电后,要求充电且能工作的时间不大于5s。\\
	\textbf{原理:}\\
	略
\item 防护等级:
	\par
	\textbf{要求:}\\
	采集单元,悬挂安装的汇集单元防护等级不低于IPX7。\\
	电杆固定安装的汇集单元防护等级不低于IPX5。\\
	\textbf{解读:}\\
	IP第一个数字表示对危险的防护,阻止人对危险物体的靠近。这里X表示没有危险。第二个数字表示对进水的防护,7表示短时浸水,5表示喷水。6表示猛烈喷水。\\
	实验的满足条件实验后,不影响正常操作,不产生破坏。水不积聚,不进入带电部分。\\
	\textbf{原理:}\\
	没什么好说的,外壳密封。

\end{enumerate}
\section{暂态录波型功能要求}
\begin{enumerate}
\item 短路和接地故障识别
	\par
	\textbf{要求:}\\
	1)应自适应负荷电流大小,当检测到电流突变且突变启动值不低于150A,突变电流持续一段时间后,各相电场强度大幅下降,且残余电流不超过5A零飘值,应能就地采集故障信息,以闪光形式就地指示故障,且能将故障信息上传至主站。\\
	2)发生接地故障,当指示器不能判断出接地故障处于安装位置上游和下游时,采集单元应能就地采集故障信息和波形,且能将故障信息和波形上传至主站进行判断,同时汇集单元应能接收主站下发的故障数据信息,采集单元以闪光形式指示故障;\\
	当指示器能够判断出接地故障处于安装位置的上游和下游时,采集单元应能就地采集故障信息和波形,以闪光形式就地指示故障,且能将故障信息和波形上传至主站。\\
	3)接地故障判别适应中性点不接地,经消弧线圈接地,经小电阻接地等配电网中性点接地方式;满足金属性接地、弧光接地、电阻接地等不同接地故障检测要求。\\
	4)当线路发生故障后,采集单元应能正确识别故障类型,并能够根据故障类型选择复位形式。
	\begin{itemize}
			\item 能识别重合闸间隔为不小于0.2s的瞬时性和永久性短路故障,并正确动作。
			\item 线路永久性故障恢复后上电自动延时复位,瞬时性故障后按设定时间或执行主站远程复位。
	\end{itemize}
	\textbf{解读1:}\\
	对于第一条,要求检测短路故障,实际同传统型一样,就是检测线路失电状况,其检测原理同上。只不过这里确定了一个参数就是$\Delta I$的定值不低于150A\\
	\textbf{原理1:}
	\footnote{$I_x$是任意相电流,$I_z$是残余电流,$U_x$是场强。}
	$$	
		\left\{ \begin{array}{ll}
				\Delta I_x>I_{dz} & \textrm{启动条件}\\
			    t>T_{dz} & \textrm{突变电流持续的时间}\\
				I_z<5A & \textrm{持续时间后,剩余电流}\\ 
			\Delta U_x>U_{dz}
		\end{array}
		\right.
	$$
	\textbf{解读2:}\\
	对于第二个要求,这里实际包括以下几个要求:1)要能够判断出接地故障。2)要能够判断出接地故障是在安装处的上游还是下游。这点供电局也不确定。3)要显示故障。4)要记录故障波形,即录波功能。5)上传主站故障波形和遥信信息。这就要求终端支持文件传输功能,上传事件功能。6)支持根据主站下发判断信息,显示故障功能(闪光)。这里包括支持命令或参数设置要求(根据规约),以及动作。\\
	3,4,5,6没有什么好说的根据规约和判断结果进行动作。重点是1,2。\\
	1)如何判断出接地故障。\\
	当前供电局接地形式包括几种:1,直接接地;直接接地一般用于110KV以上场合。2,通过电阻接地;包括小中高电阻接地;3,不接地;4,通过消弧线圈接地。在配电网中2,3,4形式都有可能。还有一种是动态电阻方式,即平时不接地,当检测到绝缘故障时,人为投入小电阻接地,以增大零序电流,利于故障检测的方式。接地故障可能有以下几种形式:1,从接地的阻值来分,可有低阻(包括金属性接地),中阻,高阻接地;2,从接地状态来分可有永久性接地,间歇性接地(包括弧光接地)。\\
	从当前继电保护原理来说接地保护包括以下几种方法:
	\begin{itemize}
			\item 零序电流保护。原理是比较稳态零序电流幅值,适用于中性点接地系统,不接地或小电阻接地系统(能可靠检测出金属性接地,低阻接地状况)。
			\item 零序电流方向保护。原理是比较稳态零序电流方向幅值,非故障线路零序电流由母线流向线路,故障线路零序电流由线路流向母线。适用于中性点接地系统,中性点不接地或小电阻接地系统(能可靠检测出金属性接地,低阻接地状况,并能判断出故障方向,从而判断出故障点位于安装位置的上游还是下游)。贾总提出的零序相位保护与此类似。
			\item 绝缘监视
	\end{itemize}




\end{enumerate}

相关文件:'F:/document/其他资料/规范性引用文件/国网规范/配电自动化/配电自动化招标技术规范/专业检测2017/故障指示器'
\end{CJK}
\end{document}

