\documentclass{article}
\usepackage{CJK}
\usepackage{amssymb}
\begin{document}
\begin{CJK}{UTF8}{gbsn}
\title{根据国网检测大纲整理的故障指示器功能}
\maketitle
根据国网检测大纲2018故障指示器的功能整理如下:
检测大纲中故障指示器按照工作原理分成两种,一种是传统型,一种是暂态录波型。

\section{传统型功能要求}
\begin{enumerate}
\item	短路故障检测
\par
	原理:
	\begin{equation}
		I_x>I_{dz}
	\end{equation}
\item	要求能够自动检测故障
\par
	原理:
	\begin{equation}	
		\left\{ \begin{array}{c}
			\Delta I_x>I_{dz} \\
			\Delta t>T_{dz} \\
			I_z<5A\\ 
			\Delta U>U_{dz}
		\end{array}
		\right .
	\end{equation}
	这里$I_x$是任意相电流,$I_z$是残余电流,U是场强。
\item	接地故障检测和告警
	\par
	要求能够根据外施加信号,暂态特征,稳态特征识别故障。\\
	外施加信号和稳态特征就是检测$I_0$的大小,暂态特征参见暂态型说明。\\
	稳态特征识别原理:
	\begin{equation}
		\left\{
			\begin{array}{ccc}
				I_0>I_{dz}\\
				\Delta t>T_{dz}
			\end{array}
			\right .
	\end{equation}
\item	复归
	\par




	
	
\end{enumerate}
\section{暂态录波型功能要求}


相关文件:'F:/document/其他资料/规范性引用文件/国网规范/配电自动化/配电自动化招标技术规范/专业检测2017/故障指示器'
\end{CJK}
\end{document}

